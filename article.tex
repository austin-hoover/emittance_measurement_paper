%% 
%% Copyright 2007-2020 Elsevier Ltd
%% 
%% This file is part of the 'Elsarticle Bundle'.
%% ---------------------------------------------
%% 
%% It may be distributed under the conditions of the LaTeX Project Public
%% License, either version 1.2 of this license or (at your option) any
%% later version.  The latest version of this license is in
%%    http://www.latex-project.org/lppl.txt
%% and version 1.2 or later is part of all distributions of LaTeX
%% version 1999/12/01 or later.
%% 
%% The list of all files belonging to the 'Elsarticle Bundle' is
%% given in the file `manifest.txt'.
%% 

%% Template article for Elsevier's document class `elsarticle'
%% with numbered style bibliographic references
%% SP 2008/03/01
%%
%% 
%%
%% $Id: elsarticle-template-num.tex 190 2020-11-23 11:12:32Z rishi $
%%
%%
\documentclass[final,5p,twocolumn,times]{elsarticle}

%% Use the option review to obtain double line spacing
%% \documentclass[authoryear,preprint,review,12pt]{elsarticle}

%% Use the options 1p,twocolumn; 3p; 3p,twocolumn; 5p; or 5p,twocolumn
%% for a journal layout:
%% \documentclass[final,1p,times]{elsarticle}
%% \documentclass[final,1p,times,twocolumn]{elsarticle}
%% \documentclass[final,3p,times]{elsarticle}
%% \documentclass[final,3p,times,twocolumn]{elsarticle}
%% \documentclass[final,5p,times]{elsarticle}
%% \documentclass[final,5p,times,twocolumn]{elsarticle}

%% For including figures, graphicx.sty has been loaded in
%% elsarticle.cls. If you prefer to use the old commands
%% please give \usepackage{epsfig}

%% The amssymb package provides various useful mathematical symbols
\usepackage{amssymb}
%% The amsthm package provides extended theorem environments
%% \usepackage{amsthm}

%% The lineno packages adds line numbers. Start line numbering with
%% \begin{linenumbers}, end it with \end{linenumbers}. Or switch it on
%% for the whole article with \linenumbers.
%% \usepackage{lineno}

\usepackage{subcaption}

\journal{Nuclear Instruments and Methods in Physics Research, A}

\begin{document}

\begin{frontmatter}

%% Title, authors and addresses

%% use the tnoteref command within \title for footnotes;
%% use the tnotetext command for theassociated footnote;
%% use the fnref command within \author or \address for footnotes;
%% use the fntext command for theassociated footnote;
%% use the corref command within \author for corresponding author footnotes;
%% use the cortext command for theassociated footnote;
%% use the ead command for the email address,
%% and the form \ead[url] for the home page:
%% \title{Title\tnoteref{label1}}
%% \tnotetext[label1]{}
%% \author{Name\corref{cor1}\fnref{label2}}
%% \ead{email address}
%% \ead[url]{home page}
%% \fntext[label2]{}
%% \cortext[cor1]{}
%% \affiliation{organization={},
%%             addressline={},
%%             city={},
%%             postcode={},
%%             state={},
%%             country={}}
%% \fntext[label3]{}

\title{Fixed-optics four-dimensional emittance measurement at the Spallation Neutron Source}

%% use optional labels to link authors explicitly to addresses:
%% \author[label1,label2]{}
%% \affiliation[label1]{organization={},
%%             addressline={},
%%             city={},
%%             postcode={},
%%             state={},
%%             country={}}
%%
%% \affiliation[label2]{organization={},
%%             addressline={},
%%             city={},
%%             postcode={},
%%             state={},
%%             country={}}

\author[author1]{A. Hoover}
\affiliation[author1]{organization={The University of Tennessee},%Department and Organization
            addressline={}, 
            city={Knoxville},
            postcode={37996}, 
            state={Tennessee},
            country={USA}}
            
\author[author2]{N. J. Evans}
\affiliation[author2]{organization={Oak Ridge National Laboratory},%Department and Organization
            addressline={One Bethel Valley Road}, 
            city={Oak Ridge},
            postcode={37831}, 
            state={Tennessee},
            country={USA}}

\begin{abstract}
Tentative: A hadron beam with a uniform charge density, elliptical transverse profile, and small four-dimensional (4D) emittance could mitigate space charge effects in circular accelerators. A scheme to approximately create such a distribution is being tested in the Spallation Neutron Source (SNS). A critical component of these efforts is to measure the 4D emittance. The 4D emittance can be reconstructed from at least four measurements of the real-space moments with different transfer matrices connecting the reconstruction and measurement locations, accomplished by varying either the machine optics (multi-optics method) or the measurement location (fixed-optics method). In this paper, we discuss the implementation of a variant of the multi-optics method using the four available wire-scanners near the SNS target, as well as the modification of the wire-scanner region to utilize the fixed-optics method — a method that is preferred due to its speed but can potentially lead to unacceptable bias and uncertainty in the reconstructed emittances. We then demonstrate the usefulness of the fixed-optics method by reconstructing the 4D emittance as a function of time during accumulation in the SNS ring for several initial experiments. 
\end{abstract}

% %%Graphical abstract
% \begin{graphicalabstract}
% %\includegraphics{grabs}
% \end{graphicalabstract}

% %%Research highlights
% \begin{highlights}
% \item Research highlight 1
% \item Research highlight 2
% \end{highlights}

% \begin{keyword}
% %% keywords here, in the form: keyword \sep keyword

% %% PACS codes here, in the form: \PACS code \sep code

% %% MSC codes here, in the form: \MSC code \sep code
% %% or \MSC[2008] code \sep code (2000 is the default)

% \end{keyword}

\end{frontmatter}

%% \linenumbers

%% main text
\section{Introduction}
\label{sec:Introduction}

Lorem ipsum dolor sit amet, consectetur adipiscing elit. Morbi cursus dapibus diam non euismod. Quisque vitae pulvinar nulla. Maecenas lobortis elementum varius. Suspendisse at purus nec massa pharetra luctus. Morbi tellus magna, aliquet a ultrices aliquet, mollis sed tortor. Nulla facilisi. Vestibulum in erat a turpis tempor porta. Nunc mi lorem, ornare eget nibh eget, lacinia posuere orci. Maecenas varius mattis nunc, ac facilisis nibh aliquet ut. Integer eu eros nulla. Etiam facilisis sapien vitae velit finibus mattis.

Donec facilisis, orci a pretium venenatis, leo metus elementum ligula, sit amet mattis sem lectus ut dolor. Fusce mollis quis nunc vel venenatis. Nulla nibh dolor, malesuada nec dapibus interdum, commodo sed velit. Maecenas feugiat nisl in rutrum bibendum. Nam mi nunc, ornare eu congue at, lacinia vitae sapien. Nullam sagittis aliquam sem, a mollis ex. Quisque blandit ex tellus, id consectetur purus pretium eget. Integer ligula urna, tincidunt ut egestas in, auctor vel neque. Etiam in nunc semper, dapibus turpis a, rhoncus nulla. Phasellus bibendum tortor eu ante vehicula volutpat. Praesent eget lorem et tellus vestibulum pharetra. Mauris lacus risus, elementum commodo magna eu, feugiat faucibus tellus. Mauris interdum, mauris in mattis ultricies, massa felis tristique sem, et ultrices lacus orci consectetur nibh.

Cras porttitor ut odio vel pharetra. Proin porttitor quis mi a ultrices. Morbi egestas neque at vestibulum tristique. Praesent consequat enim eget nulla condimentum vestibulum ac sed turpis. Suspendisse libero ex, mollis eget erat et, pretium iaculis sapien. Nulla ultricies feugiat sapien non pellentesque. Duis tortor felis, ullamcorper a porttitor sit amet, tempor quis lorem. Maecenas blandit faucibus dolor, eu laoreet libero interdum non. Aliquam interdum ante lacus, vitae porta nunc ultricies id. Fusce porta interdum leo a bibendum. Proin tristique purus et bibendum rutrum. Duis molestie vulputate urna, vitae interdum diam sodales nec. Nunc pulvinar purus erat, sit amet consectetur magna mattis eu. Etiam tempus malesuada feugiat. Vivamus vulputate nunc ut eros imperdiet lobortis.

Interdum et malesuada fames ac ante ipsum primis in faucibus. Vivamus ultricies felis blandit dolor vestibulum semper. Nunc ultricies erat magna, id laoreet nisl maximus maximus. Etiam non tortor lectus. Suspendisse sodales mi elit, non malesuada orci condimentum ac. Sed tincidunt, urna eget facilisis ultricies, nulla diam laoreet nibh, sit amet ullamcorper eros leo faucibus nisi. Fusce eget vestibulum odio. Integer hendrerit erat sit amet tincidunt consequat. Quisque magna metus, iaculis ut tortor id, euismod posuere urna. Quisque a sapien tincidunt, laoreet eros eget, interdum lectus. Fusce ac faucibus felis. Praesent viverra lorem in porttitor viverra. Ut consectetur congue risus, vitae venenatis sapien imperdiet in. Sed rutrum, nulla ut ultrices suscipit, nulla lacus dapibus mi, id viverra ante velit nec mi.

Vestibulum a turpis consequat, imperdiet lectus nec, gravida est. Morbi ut mi tempus tellus maximus ullamcorper at ut sem. Morbi dolor libero, finibus at porta a, suscipit a ex. Integer enim lectus, auctor a posuere commodo, egestas pharetra ipsum. Aliquam sit amet cursus risus. Praesent sollicitudin neque vitae mattis dignissim. Maecenas metus elit, molestie non fermentum a, iaculis sit amet diam. Nullam molestie aliquet sapien, hendrerit congue lacus. Sed nec nunc at tellus tincidunt scelerisque nec in turpis. Nam sed vulputate sem. Integer accumsan, odio ut vestibulum vehicula, tortor tellus sagittis nibh, eleifend finibus quam quam sagittis dui. Cras mollis diam nec lectus fringilla efficitur. \cite{Xiao2013}




\section{Methods}
\label{sec:Methods}


\begin{figure}
    \centering
    \includegraphics[width=\columnwidth]{figures/multi_optics_measurement.pdf}
    % \par\medskip
    \begin{tabular}{lll}
        \small\textbf{Parameter} & \small\textbf{Measurement} & \small\textbf{Model} \\
        % \midrule
        \small$\beta_x$ [m/rad] & \small22.06 $\pm$ 0.29 & \small22.00 \\
        \small$\beta_y$ [m/rad] & \small4.01 $\pm$ 0.02 & \small3.81 \\
        \small$\alpha_x$ & \small2.33 $\pm$ 0.04 & \small2.37 \\
        \small$\alpha_y$ & \small-0.49 $\pm$ 0.01 & \small-0.60 \\
        \small$\varepsilon_1$ [mm~mrad] & \small33.02 $\pm$ \small0.05 & - \\
        \small$\varepsilon_2$ [mm~mrad] & \small25.67 $\pm$ \small1.03 & - \\
        \small$\varepsilon_x$ [mm~mrad] & \small32.85 $\pm$ \small0.05 & - \\
        \small$\varepsilon_y$ [mm~mrad] & \small25.87 $\pm$ \small0.12 & - \\
      \end{tabular}
    % \par\medskip
    \caption{Reconstructed beam parameters and graphical output (in normalized phase space) from a multi-optics emittance measurement of a production beam.}
    \label{fig:multi_optics_measurement}
\end{figure}


\begin{figure}
    \centering
    \includegraphics[width=\columnwidth]{figures/sensitivity_scan.pdf}
    \caption{Scan of the phase advances at WS24.}
    \label{fig:sensitivity_scan}
\end{figure}


\begin{figure*}
    \centering
    \begin{subfigure}{0.95\columnwidth}
        \includegraphics[width=\textwidth]{figures/mismatch.pdf}
        \caption{}
        \label{fig:mismatch}
    \end{subfigure}
    \hspace{1cm}
    \begin{subfigure}{0.95\columnwidth}
        \includegraphics[width=\textwidth]{figures/mismatch_unequal_emittances.pdf}
        \caption{}
        \label{fig:mismatch_unequal_emittances}
    \end{subfigure}
    \caption{Simulated fractional error and standard deviation of reconstructed intrinsic emittances as a function of the beam mismatch parameter. (a) $\varepsilon_1 / \varepsilon_2 = 1$; (b) $\varepsilon_1 / \varepsilon_2 = 3$.}
\end{figure*}


\section{Example application}
\label{sec:Example application}


\begin{figure}
    \centering
    \begin{subfigure}{0.49\columnwidth}
        \includegraphics[width=\textwidth]{figures/emittances_exp1a.pdf}
    \end{subfigure}
    \begin{subfigure}{0.49\columnwidth}
        \includegraphics[width=\textwidth]{figures/emittances_exp1b.pdf}
    \end{subfigure}
    \caption{Caption.}
    \label{fig:mismatch}
\end{figure}


\section{Conclusion}
\label{sec:Conclusion}

This is a sentence to fill some space.



%% The Appendices part is started with the command \appendix;
%% appendix sections are then done as normal sections
%% \appendix

%% \section{}
%% \label{}

%% If you have bibdatabase file and want bibtex to generate the
%% bibitems, please use
%%
%%  \bibliographystyle{elsarticle-num} 
%%  \bibliography{<your bibdatabase>}

%% else use the following coding to input the bibitems directly in the
%% TeX file.

\bibliographystyle{elsarticle-num} 
\bibliography{bibliography.bib}


% \begin{thebibliography}{00}

% %% \bibitem{label}
% %% Text of bibliographic item

% \bibitem{}

% \end{thebibliography}



\end{document}
\endinput
%%
%% End of file `elsarticle-template-num.tex'.
